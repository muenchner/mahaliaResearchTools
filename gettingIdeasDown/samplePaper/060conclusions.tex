We have presented a computationally efficient method for selecting a subset of damage maps, corresponding ground-motion intensity maps, and associated occurrence rates for a probabilistic infrastructure network risk assessment using optimization.
%significance/advance
%The approach introduces a constraint on the exceedance rate consistency of a performance measure when selecting a subset of ground-motion intensity maps. This implicitly captures the joint distribution of the spectral acceleration in the optimization objective function.
Through a case study of the San Francisco Bay Area road network, we have demonstrated how to use the optimization formulation to select a subset of damage maps (with corresponding ground-motion intensity maps and occurrence rates) from a larger set of candidate maps. The problem minimizes the error in estimating the marginal distributions of ground-motion intensity at individual locations as well as the distribution of a proxy network performance measure (percentage of bridges damaged).  The proxy performance measure implicitly captures the joint distribution of the spectral acceleration in the optimization objective function. Furthermore, we describe checks of consistency with the ground motion hazard such as fault distribution and discuss the error in estimating the exceedance curve of the proxy performance measure and of the target performance measure (percentage change in average morning commute time). We have shown that the results from the subset are a good estimate of the results from an extensively-sampled baseline set of maps. 
%more significance
Thus, the researcher or decision maker can estimate the exceedance rates of a target performance measure with a reduced subset of damage maps while still achieving reasonable accuracy. This is significant, because many performance measures are extremely computationally expensive to evaluate. Thus, this work can be used to transform a ``what-if'' scenario approach into one using an event-based probabilistic loss estimation model for assessing risk efficiently. With this improved understanding, researchers and policy makers can better mitigate risks and increase community  resiliency. 
Additionally, this work aids emergency response planning by identifying a small, but representative, set of scenarios to consider in planning exercises.
%Another significance of this work is emergency response planning. By identifying a small number of scenarios corresponding to a range of network performance outcomes, this method may improve emergency response planning by pointing out scenarios to simulate for emergency response preparedness. For example, in the final set of 25 maps in the case study, three scenarios at equal intervals of traffic network performance are the less disruptive M6.45 Calaveras fault event, moderately disruptive M7.15 Calaveras fault event, and extremely disruptive M8.05 N. San Andreas fault event. %6.95 Hayward Rogers-Creek
%applications
Finally, this approach can potentially be applied to efficiently analyze risk from other natural hazards impacting networks.


\acks
We thank Dave Ory at the Metropolitan Transportation Commission (MTC), and Tom Shantz and Loren Turner at Caltrans for motivating discussions and providing the case study network data. We also gratefully acknowledge Madeleine Udell of Stanford University for her help with the condensed optimization formulation and discussion in Section 3; the manuscript is greatly improved based on her feedback. We also thank Jessica Jacobo of the University of California-Berkeley for painstakingly integrating the Caltrans and MTC data. She reconciled the approximate geographic coordinates provided for bridges with the 3D locations of the road segments, using estimated geographic coordinates of the road segments, text descriptions, and aerial images. Furthermore, thanks to the anonymous reviewers for helpful feedback. MTC develops and maintains a regional travel model for regional planning purposes.  As a public agency, the MTC sees the model as a public good and welcomes its use by others.  However, the authors are entirely responsible for the use of the model in the present context, including assessing the reasonableness of the results for the task at hand.  MTC has not reviewed and does not endorse the model results, implicitly or otherwise. Also, the first author gratefully acknowledges the support of the Stanford Graduate Fellowship and the National Science Foundation Graduation Research Fellowship. This work was supported in part by the National Science Foundation under NSF grant number CMMI 0952402. Any opinions, findings and conclusions or recommendations expressed in this material are those of the authors and do not necessarily reflect the views of the National Science Foundation. 